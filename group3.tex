\documentclass[a4paper]{article}
\usepackage{amsmath}
\usepackage{graphicx}
\usepackage{geometry}
\usepackage{floatrow}
\usepackage{layout}
\usepackage{amssymb} 
\usepackage{multirow}
\usepackage[utf8]{inputenc}
\usepackage{caption}
\geometry{margin=1in}
\usepackage{authblk}
\usepackage{indentfirst}


\begin{document}


\title{\textbf{\huge{ENS210 Project Proposal}}}
\author{\textbf\large{Berra Say{\i}n \vspace{3ex} Fatih Ta\c{s}yaran \vspace{3ex} Rana Kalkan} \\ \vspace{-4ex} \textit{Instructor: Ogün Adebali}}
%\affiliation{The	College,	University	of	Chicago,	Chicago,	Illinois	60637,	USA}
%\affil{\textbf{Sichuan University,	Chengdu,		China}}
\date{\today}
%	Abstract
\maketitle
%\begin{thebibliography}{10}
%\end{thebibliography}

Characterizing a protein family in an evolutionary aspect could give hints about function, evolutionary history, critical residues and domains of this protein family. In order to observe a protein's functionality across animals which live in aquatic and terrestrial environments and also animals like tetrapods who live in intersection of this environments, we chose the protein family \textbf{Carbamoyl-phosphate synthase, large subunit (CPS)}.
\par
\smallskip
In general, animals who live in different environments have different excretory systems. However, animals who live in the same environment and who are different kind of species can also have different excretory system. By examining the protein family which is one of the families that differ within different excretory systems, we are aiming to find evolutionary history of this protein family and it's effect on the phenotypes of different species.
\par
\smallskip
Ammonia is poisonous to all vertebrate, although it is not dangerous in low density and a good solvent in water. Therefore, it is possible to excrete ammonia for animals who  live in abundance of water called aquatic animals. In contrary, there is no such feasibility for terrestrial animals so that they have to convert ammonia into urea or uric acid in their livers by reacting ammonia with $CO_2$ via the urea cycle and excrete urea with consuming more energy due to this process.
\par
\smallskip
As mentioned before, species and their excretory systems are intervening in terms of environments. Whales who are mammals excrete urea in contrary to marine fishes. But some fishes which live in very alkaline environments excrete urea in contrary to marine fishes. Also there is the case of animals like amphibians, which live in both of these two environments, demonstrates three of this nitrogen waste forms. Because of this variety, we want to investigate the evolutionary history of the protein family of \textbf{CPS} that has a significant role in this process. 

\end{document}
